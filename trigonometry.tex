\documentclass{article}
\usepackage{amsmath}
\usepackage{amssymb}
\usepackage{mathtools}

\begin{document}

Identities derived through a parallelogram diagram on two concentric circles.
\begin{align*}
    \cos\theta + r\cos(\theta + \alpha) = \sqrt{1 + 2r\cos\alpha + r^2} \cos\left(\theta + \cos^{-1}\left(\frac{1+r\cos\alpha}{\sqrt{1+2r\cos\alpha + r^2}}\right)\right)
\end{align*}

\begin{align*}
    cos\theta + r\cos(\theta + \alpha) = \left(1 + r\cos\alpha\right)\cos\theta - r\sin\alpha\sin\theta
\end{align*}
which implies a more fundamental trig identity,
\begin{align*}
    cos\left(\theta + \alpha\right) = \cos\alpha\cos\theta - \sin\alpha \sin\theta.
\end{align*}
(Angle sum formulas are more easily derived by composition of rotation matrices.)


\end{document}

